\chapter{Conclusion and Recommendation}
\label{ch_6}
\hspace{\parindent} 

In this work, we have completely defined the hypergeometric-type series containing the digamma function as a factor that naturally arises in the calculation of the finite-part integral in the case of pole singularity. The series is defined as the function ${}_p\tilde{\psi}_q$ and analytically continue its analyticity to the whole complex plane. Along the way, we set up a relation of this defined function ${}_p\tilde{\psi}_q$ to Kampé de Fériet function. The relation was successfully used in the results of finite-part integration to come up with a specific values and reduction formulas of the Kampé de Fériet function. In fact, using the method, we can already produced a lot of reduction formula of the Kampé de Fériet function from the tabulated finite-part integration and integral representation of the generalized hypergeometric function
\begin{equation}\label{general}
\, _{p+1}F_{q+1}\!\left.\left(\begin{array}{c}
\beta,\alpha_p\\
\beta+\alpha,\rho_q
\end{array}\right|z\right) = \frac{\Gamma(\beta+\sigma)}{\Gamma(\beta)\Gamma(\sigma)} \int_0^{\infty} \frac{s^{\beta-1}}{(s+1)^{\beta+\sigma}} \, _{p}F_{q}\!\left.\left(\begin{array}{c}
\alpha_p\\
\rho_q
\end{array}\right|\frac{z s}{s+1}\right)\mathrm{d}s .
\end{equation}
The obstacle needed to overcome in performing the method of finite-part integration is the evaluation of the finite-part integral that will arise. This step is very crucial since it is where the function ${}_p\tilde{\psi}_q$ appears. 

Currently, the evaluation of the finite-part integral is only limited to its canonical definition. This limitation serves as the constraints to the method applications. It serves as a call to develop techniques for the evaluation of finite-part integral similar to the established methods for the evaluation of convergent integral.  We expect complementary techniques in separating the finite part from the divergent integral without explicitly referencing the canonical definition of finite-part integral. 
