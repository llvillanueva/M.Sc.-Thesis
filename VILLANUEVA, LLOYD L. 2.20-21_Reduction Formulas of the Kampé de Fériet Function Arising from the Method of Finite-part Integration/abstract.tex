%Finite-part integration is a method of evaluating well-defined convergent integrals by means of the finite-part of divergent integrals. The method is known to exactly evaluate the Stieltjes transform $S_{\lambda} [f] = \int_0^\infty f(x)/(\omega + x)^{\lambda} \, \mathrm{d}x$ with the assumption that the complex extension of the function $f(x)$ is entire. The recent development of the method is the extension of its applicability to the Stieltjes transform of functions $f(x)$ with competing singularities in its complex extension. Also, the presence of singularities was found to impose a restriction to the radius of convergence of the infinite series of finite-part integrals. This infinite series of the finite-part integrals is evaluated to be the hypergeometric-type series containing digamma function as a factor in the case of pole singularity. Thus, the study of this series is necessary to lift this restriction. In this thesis, the convergence and analyticity of the generalized form of the mentioned series will be established. Then, the relation of this series to the Kampé de Fériet function will be derived. Furthermore, the obtained relation will be exploited, together with the results in the method of finite-part integration, to generate reduction formulas of the Kampé de Fériet function.
Finite-part integration is a method of evaluating well-defined convergent integrals by means of the finite-part of divergent integrals. The method is known to exactly evaluate the Stieltjes transform $S_{\lambda} [f] = \int_0^\infty f(x)/(\omega + x)^{\lambda} \, \mathrm{d}x$ with the assumption that the complex extension of the function $f(x)$ is entire. The recent development of the method is the extension of its applicability to the Stieltjes transform of functions $f(x)$ with competing singularities in its complex extension. Also, the presence of singularities was found to impose a restriction to the radius of convergence of the infinite series of finite-part integrals. In the case of pole singularity, the infinite series of the finite-part integrals takes the form of the hypergeometric-type series containing digamma function as a factor when evaluated. Thus, in order to lift the restriction, the study of the mentioned series is necessary. In this thesis, the convergence and analyticity of the generalized form of the mentioned series will be established. Then, the relation of this series to the Kampé de Fériet function will be derived. Furthermore, the obtained relation will be exploited, together with the results in the method of finite-part integration, to generate reduction formulas of the Kampé de Fériet function.

\noindent PACS: 02.30.Uu (Integral transforms); 02.30.Gp (Special functions); 02.30.Cj (Measure and integration)