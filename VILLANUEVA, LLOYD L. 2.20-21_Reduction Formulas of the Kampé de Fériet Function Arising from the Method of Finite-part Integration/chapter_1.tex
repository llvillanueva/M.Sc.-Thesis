\chapter{Introduction}

Special functions often arise as a solution to a differential equations. In physics, most of the physical laws are expressed in the form of differential equation. Thus, the study of special function and its transformation, identities, and reduction formula is an active research topic \cite{bateman1953higher, nikiforov1988special,  abramowitz1964handbook, gradshteyn2014table}. One of the most known special function is the hypergeometric function, which has been a useful tool in the field of analysis. The generalized hypergeometric function is expressed as 
\begin{equation} \label{GHF}
    \pFq{p}{q}{a_1, a_2, ..., a_p}{b_1, b_2, ..., b_q}{z} = \sum_{k=0}^{\infty} \frac{(a_1)_k \, (a_2)_k \, ... \, (a_p)_k}{(b_1)_k \, (b_2)_k \, ... \, (b_q)_k} \frac{z^k}{k!},
\end{equation}
where $(a)_k = \Gamma(k+a)/\Gamma(a)$ is a Pochhammer symbol and $z$ is a complex number \cite{slater1966generalized}. The great accomplishments of the theory of hypergeometric function in one variable lead to the establishment of a corresponding theory in two or more variables. 

The hypergeometric function in two variables was first introduced by Appell \cite{appell1880classe}. In his paper, he presented four possible combinations of the double hypergeometric series $F_1, F_2, F_3, F_4$. Those series are referred to in the literature as the Appell series. Next, Horn completed the set of all possible combinations of the second-order hypergeometric series in two variables and denoted them as $G_1, G_2, G_3, H_1, \dots, H_7$ \cite{horn1931hypergeometrische}, which is referred to in the literature as the Horn series. Then, he also defined a confluent form of the Horn series and Borngässer completed them \cite{borngasser1933uber}. For the confluent form of the Appell series, Humbert established them in \cite{humbert1922ix}. %while Appell and Kampé de Fériet described them reasonably fully%. 
To unify the four Appell series, Kampé de Fériet initiated the generalization of the hypergeometric series in two variables \cite{appell1926fonctions}. Burchnall and Chaundy shortened the notation used by Kampé de Fériet for his double hypergeometric series of superior order \cite{burchnall1941expansions}. Using the notation introduced in \cite{srivastava1976integral}, the more generalized form of hypergeometric series in two variables is expressed as 
\begin{align}
\begin{split} \label{kampein}
    & F^{p:\;q;\;k}_{l:\;m;\;n}\left[\begin{array}{ccc}
         (a_p):& (b_q); & (c_k)  \\
         (\alpha_l):& (\beta_m); & (\gamma_n) 
    \end{array};\;  x,y\right] 
    \\& \hspace{20mm} =  \sum_{r=0}^{\infty}\sum_{s=0}^{\infty} \frac{\prod_{j=1}^p (a_j)_{r+s} \prod_{j=1}^q (b_j)_r \prod_{j=1}^k (c_j)_s}{\prod_{j=1}^l (\alpha_j)_{r+s} \prod_{j=1}^m (\beta_j)_r \prod_{j=1}^n (\gamma_j)_s} \frac{x^r}{r!}\frac{y^s}{s!},
\end{split}
\end{align}
where $x,y$ are complex numbers and $(a_j)$ stands for the sequence of $j$ parameters $a_1, ..., a_j$. The said series converges under the following conditions: ({\it i}) $p+q < l+m+1$, $p+k < l+n+1$, and $|x| < \infty$, $|y| < \infty$. ({\it ii}) $p+q = l+m+1$, $p+k = l+n+1$ and $|x|^{1/(p-l)} + |y|^{1/(p-l)} < 1$ if $p > l$, $\mbox{max}(|x|,|y|)<1$ if $p\leq l$ \cite{srivastava1976integral}. Even though the series introduced by Kampé de Fériet is only a special case of the double hypergeometric series defined in equation \eqref{kampein}, the literature still referred to it as the Kampé de Fériet series or function.  This function represents a solution to a wide range of problems in pure and applied mathematics and mathematical physics \cite{alder1956study, reynolds1964some,exton1976multiple, exton1978handbook, ancarani2010derivatives}. Moreover, identifying reduction formula of the Kampé de Fériet function or reduction of this function into one variable i.e. $z=x=y$  have a great value in simplifying solutions \cite{cvijovic2010reduction}. Therefore, compilations of reduction formulas \cite{miller2006summations, srivastava1985multiple, exton1998register} are important since there is no priori way of knowing their existence. 

Recently in \cite{SPP-2020-2G-03}, a reduction formula of Kampé de Fériet function was successfully obtained by performing finite-part integration \cite{galapon2017problem} to a known Stieltjes transform \cite{saxena1959study}. Finite-part integration is a method of evaluating well-defined convergent integral by utilizing the finite part of the divergent integral. The method is first established to resolve the problem of missing terms arising from term by term integration involving divergent integrals in the standard Stieltjes transform \cite{mcclure2016explicit}. The method resolved the problem of missing terms by lifting the integration in the complex plane, and showed that the missing terms come from the singularities of the kernel of the Stieltjes transform when the divergent integral arising from the term by term integration is interpreted as a finite-part integrals. This finite-part integral is equivalent to Cauchy principal value \cite{pipkin1991course} and Hadamard finite part \cite{hadamard1923lectures} but in different representation \cite{galapon2016cauchy}. In fact, there are a lot of ways to assign a value to the divergent integral, namely,  analytic continuation \cite{RevModPhys.47.849},  distribution theory \cite{wong2016distributional}, regularization method \cite{shiekh1990zeta} and others \cite{laforgia2009theory, caianiello1973generalized, costin2014foundational}. Those techniques on giving meaning with divergences exists since we are forced to deal with them, especially when they appear in a certain physical phenomenon \cite{bonnet1999boundary, frankel2006generalizing, frankel2007regularization}. However, all of these techniques cannot be use immediately on the problem of missing terms. The establishment of the calculus in each techniques is necessary to ensure the recovery of the missing terms. As of the moment, only distribution theory of Wong \cite{wong2016distributional} and finite-part integral of Galapon \cite{galapon2017problem} have a developed calculus.

%The finite part in the method is the same as the finite part introduced by Hadamard \cite{hadamard1923lectures}. However in the method, we are only restricted to the function 

%This method is also known to resolve the problem of missing terms arising from term by term integration involving divergent integrals in the standard Stieltjes transform \cite{mcclure2016explicit}. In the method, the Stieltjes transform was lifted to the complex plane and showed that the   

%The divergent integral is induced by naively performing a term by term integration. 


%The method is first established to answer the problem of missing terms arising from term by term integration involving divergent integrals in the standard Stieltjes transform \cite{}. The problem of missing terms is resolved by lifting the integration in the complex plane. It was shown that the missing terms come from the singularities of the kernel of the Stieltjes transform when the divergent integral arising from the term by term integration is interpreted as a finite-part integrals. This finite-part integrals 

%The divergent integrals is induced by naively performing a term by term integration. This term by term integration involving divergent integrals may result to a missing terms \cite{mcclure2016explicit}. However, the problem is resolve through the method of finite-p


%The method of finite-part integration \cite{galapon2017problem} offers a new way of deriving novel and known transformations, identities, and reduction formulas of special functions specifically by exploiting the their known integral transformation  \cite{ticathetit,tica2018finite,tica2019finite,SPP-2018-PC-41,SPP-2019-PB-05,Villanueva_Galapon_2019,SPP-2020-2G-03,doi:10.1063/5.0038274}. It is a method of evaluating well-defined convergent integral by utilizing the finite-part of the divergent integral. In the method, the value of the finite part of the divergent integral is assigned to Analytic Principal Value (APV) \cite{galapon2016cauchy}. The APV is an equivalent but a distinct representation of the Cauchy principal value \cite{pipkin1991course} and Hadamard finite-part \cite{hadamard1923lectures}. This alternative definition of the Hadamard finite-part integral gives way to the creation of the method. 

Currently, the method of finite-part integration was successfully implemented to exactly and asymptotically evaluate the generalized Stieltjes transform of order $\lambda$ of the function $f(x)$ that is locally integrable in the interval $[0,\infty)$
\begin{equation} \label{1.1}
    S_\lambda[f] = \int_0^\infty \frac{f(x)}{(\omega+x)^\lambda} \mathrm{d}x, \quad 0 < \omega < \infty,
\end{equation}
where the integral exists in the Riemann sense. The application of the method to known Stieltjes integral representations of some special functions led to a new representation of them \cite{SPP-2018-PC-41,Villanueva_Galapon_2019, SPP-2019-PB-05}. In the papers \cite{tica2018finite, tica2019finite}, the generalized Stieltjes transform of $f(x)$ with an entire complex extension was exactly and asymptotically evaluated using the method of finite-part integration. Recent development of the method was completely discuss in \cite{doi:10.1063/5.0038274}, where the application of the method extends further to the function $f(x)$ with a competing singularity to its complex extension. The results of this new applicability requires us to further study the summation involving digamma function since its often appears in the calculation of the finite-part integral in the case of pole singularity.

In this thesis, we will demonstrate the method of finite-part integration to evaluate a Stieltjes transform and show how the hypergeometric-type series containing digamma function naturally arises in the calculation of finite-part integral. We will define this series as the function 
\begin{equation} 
    \pPq{p}{q}{\Vec{a}}{\Vec{b}}{c}{z} =  \sum_{k=0}^{\infty} \frac{\prod_{l=1}^{p} (a_{l})_k}{ \prod_{l=1}^{q} (b_{l})_k} \psi(k+c) \frac{z^k}{k!},
\end{equation}
and establish its convergence. Next, we will determine the nature of the singularities of this function and analytically continue it to the whole complex plane. Then, we will derive a relation of this function to the Kampé de Fériet function reduction formula and exploit the relationship together with the results on the method of finite-part integration to tabulate reduction formulas of the Kampé de Fériet function. In contrast to \cite{karlsson1983some, cvijovic2008closed,choi2019general,ali2013some,article333}, where the reduction formulas of the Kampé de Fériet function were derived through the exploitation of the theorems and identities of double sum, in this work, the reduction formulas of the Kampé de Fériet function arises naturally in the method.

The rest of the thesis is organized as follows: In Chapter \ref{ch_3}, we fully discuss the method of finite-part integration. In Chapter \ref{ch_4}, we define the hypergeometric-type series containing the digamma function as the ${}_p\tilde{\psi}_{q}$ function, extend its analyticity, and derive a relation to the Kampé de Fériet function. In Chapter \ref{ch_5}, we will use the method of finite-part integration and the derive relation from the previous chapter to generate reduction formulas of Kampé de Fériet function. The conclusion and recommendations is written in Chapter \ref{ch_6}.